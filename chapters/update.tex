

\section{未完待续}

目前该模板基本可以应付日常论文写作需要,\par
尤其是对于我航学子,你们看看这个模板,是不是似曾相识,(尤其是能不能过冯如杯格式审查)。\par
限于精力,更多高级功能,请待作者再择良辰,Someday有朝一日还会回来。

\section{模板更新记录}

经123学长指正,Someday于2017.09.08进行一次重要更新,内容包括:\par
1、对中文字号的设置命令进行了修改,使用方法不变。\par
2、对仿宋字体所在的fontstyle进行了修改,以适应fontspec宏包。\par
3、将取消首页页码的命令改为:pagenumbering\{gobble\} \par

以上修正解决了以前版本中编译报错问题,当前版本在TeXLive2016环境下已经可以一次编译成功。 \par

编译命令如下: \par
cd\ <模板根目录> \par
xelatex main.tex \par

\par \ 
\par \ 

\section{模板新的更新记录}

lawye和nikkukun于2019.4.11开始维护该模板.

本次更新基于\it{第二十九届“冯如杯”学生学术科技作品竞赛论文撰写格式规范}\cite{格式}.
\par \

\it{GhostNet}\cite{ghostnet}

\subsection{2019.4.11日}
\begin{enumerate}
    \item 修改参考文献格式
    \item 修改了字号和页眉

\end{enumerate}

\subsection{2019.4.16日}
\begin{enumerate}
    \item 修改目录与标题的行间距

\end{enumerate}

Pannenets.F 在2020.4.10出于个人需要修改了本模板.

\subsection{2020.4.10}
\begin{enumerate}
    \item 增加了副标题对华文新魏的支持
    \item 增加了对外文参考文献\cite{ghostnet}出现无出版地点s:l, s:n的修正,直接复用了http://haixing-hu.github.io/nju-thesis/提供的bst文件
    \item 重新组织章节分为各个小chap.tex
    \item 增加了对结语的支持
\end{enumerate}


\subsection{2020.4.11}
\begin{enumerate}
    \item 修改了图表标题为宋体加粗
\end{enumerate}

\subsection{2020.4.18}
\begin{enumerate}
    \item 改正了标题格式以及目录格式
\end{enumerate}

\subsection{2020.4.20}
\begin{enumerate}
    \item 改正了默认图片格式防止错位
\end{enumerate}

注:第二十九届与第三十届格式无(肉眼可见)的区别

